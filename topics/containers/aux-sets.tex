\section{Sets}

\frame{\tableofcontents[currentsection]}

\begin{frame}
  \frametitle{Sets}
  \begin{itemize}
    \item Bij lists is het opzoeken van elementen traag
    \item Sets specialiseren zich in het snel opzoeken van elementen
    \item Een set bepaalt intern zelf de volgorde van de elementen
    \item Set positioneert elementen op slimme manier
    \item Volgorde zo gekozen om elementen snel terug te vinden
          \begin{itemize}
            \item Bv.~alfabetisch ordenen kan opzoeken sterk versnellen
            \item Hashing, nog sneller dan alfabetisch
          \end{itemize}
  \end{itemize}
\end{frame}

\begin{frame}
  \frametitle{Sets vs Lists}
  \begin{itemize}
    \item Positie in de set is ``geheim''
    \item Je kan geen element op een bepaalde positie plaatsen
    \item Je kan niet vragen waar een element zit
    \item Je mag enkel vragen ``zit dit element \emph{ergens} de set?''
  \end{itemize}
  \begin{center}
    \begin{tabular}{cc}
      list & set \\
      \toprule
      \begin{tikzpicture}
        \path[use as bounding box] (0,-1) rectangle (5,1);
        \foreach \x in {1,2,3,4} {
          \node[fill=red!50,draw,drop shadow,minimum size=1cm] at (\x,0) {\x};
        }
      \end{tikzpicture}
      &
      \begin{tikzpicture}
        \draw[fill=red!50,drop shadow] (0,0) ellipse [x radius=2cm,y radius=1cm];
        \draw[fill=black] (-1,0) circle [radius=0.05cm] node[above] {1};
        \draw[fill=black] (0.75,0.25) circle [radius=0.05cm] node[above] {2};
        \draw[fill=black] (0,0) circle [radius=0.05cm] node[above] {3};
        \draw[fill=black] (0.25,-0.5) circle [radius=0.05cm] node[above] {4};
      \end{tikzpicture}
    \end{tabular}
  \end{center}
  \code[language=python,width=.5\linewidth] {set.py}
\end{frame}

\begin{frame}
  \frametitle{Sets vs Lists}
  \begin{itemize}
    \item Twee lijsten zijn gelijk indien ze dezelfde elementen op dezelfde plaats bevatten
    \item Sets kennen geen positie toe
    \item Volgorde kan dus ook niet van belang zijn
    \item Twee sets zijn gelijk indien ze dezelfde elementen bevatten
  \end{itemize}
  \code[language=python]{set-equality.py}
\end{frame}

\begin{frame}
  \frametitle{Sets vs Lists}
  \structure{Meermaals Voorkomend Element}
  \begin{itemize}
    \item Een set kan een element niet meer dan 1$\times$ bevatten
    \item Element 2de maal toevoegen heeft geen effect
  \end{itemize}
  \code[language=python]{set-multiple.py}
  \vskip2mm
  \structure{Effici\"entie}
  \begin{itemize}
    \item Toevoegen element gaat sneller dan bij lists
    \item Verwijderen element sneller dan bij lists
    \item ``Is element van'' sneller dan bij lists
  \end{itemize}
\end{frame}

\begin{frame}
  \frametitle{Operaties op Lists}
  \begin{itemize}
    \item Toevoegen: \texttt{xs.add(5)}
    \item Verwijderen: \texttt{xs.remove(1)}
    \item Unie: \texttt{xs.union(ys)}
    \item Doorsnede: \texttt{xs.intersection(ys)}
    \item Deelverzameling: \texttt{xs.subsetof(ys)}
    \item \dots
  \end{itemize}
\end{frame}

%%% Local Variables:
%%% mode: latex
%%% TeX-master: "containers"
%%% End:

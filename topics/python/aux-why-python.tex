\section{Waarom Python?}

\subsection{Wie Gebruikt Python?}

\frame{\tableofcontents[currentsubsection]}


\begin{frame}
  \frametitle{Rankings}
  \begin{columns}
    \column{5cm}
    \begin{center}
      \begin{tabular}{cc}
        \textbf{Ranking} & \textbf{Taal} \\
        \toprule
        1 & Java \\
        2 & C \\
        3 & C++ \\
        4 & C$^\sharp$ \\
        5 & Python \\
      \end{tabular}
      \vskip5mm
      \link{http://www.tiobe.com/tiobe-index/}{\tiny TIOBE Index for January 2017}
    \end{center}
    \column{5cm}
    \begin{center}
      \begin{tabular}{cc}
        \textbf{Ranking} & \textbf{Taal} \\
        \toprule
        1 & JavaScript \\
        2 & Java \\
        3 & Python \\
        4 & CSS \\
        5 & PHP \\
      \end{tabular}
      \vskip5mm
      \link{http://githut.info/}{\tiny githut.info}
    \end{center}
  \end{columns}
  \vskip5mm
  \begin{quote}
    Eight out of the top 10 universities now use Python to introduce programming --- \link{http://www.javaworld.com/article/2452940/learn-java/python-bumps-off-java-as-top-learning-language.html}{Javaworld}
  \end{quote}
\end{frame}


\begin{frame}
  \frametitle{Gebruik}
  \begin{center}
    \begin{tikzpicture}
      \node at (2, 3) {\includegraphics[width=2cm]{google.png}};
      \node at (-2, 0) {\includegraphics[width=3cm]{dropbox.png}};
      \node at (4, -1) {\includegraphics[width=1cm]{facebook.png}};
      \node at (-1, 2) {\includegraphics[width=2cm]{quora.png}};
      \node at (3, 1) {\includegraphics[width=3cm]{reddit.png}};
      \node at (-2, -2) {\includegraphics[width=2cm]{pinterest.png}};
      \node at (1, -2) {\includegraphics[width=2cm]{youtube.png}};
    \end{tikzpicture}
  \end{center}
\end{frame}

\subsection{Eenvoud}

\frame{\tableofcontents[currentsubsection]}

\begin{frame}
  \frametitle{Hello World!}
  \begin{itemize}
    \item Traditie gebiedt dat het eerste programma dat je schrijft
          in een nieuwe taal ``Hello World'' is
    \item Dit programma moet ``Hello World'' afprinten
  \end{itemize}
\end{frame}

\begin{frame}
  \frametitle{Java}
  \code[language=java,font=\small,width=.9\linewidth]{hello-world.java}
  \begin{tikzpicture}[overlay,remember picture,
                      message/.style={opacity=0.75,text opacity=1.0,drop shadow,fill=blue!50,font=\small},
                      link/.style={-latex}]

    \onslide<2->{
      \codeoverlinex{class}
      \node[message] (message) at ($ (class) + (0,1) $) {Klassedeclaratie};
      \draw[link] (message) -- (class);
    }
    \onslide<3->{
      \codeoverlinex{public}
      \node[message] (message) at ($ (public) + (-1,1) $) {Access modifier};
      \draw[link] (message) -- (public);
    }
    \onslide<4->{
      \codeoverlinex{static}
      \node[message] (message) at ($ (static) + (1,1) $) {Klassemethodes};
      \draw[link] (message) -- (static);
    }
    \onslide<5->{
      \codeoverlinex{void}
      \node[message] (message) at ($ (void) + (1,2) $) {Returntypes};
      \draw[link] (message) -- (void);
    }
    \onslide<6->{
      \codeoverlinex{main}
      \node[message] (message) at ($ (main) + (1.5,1) $) {Entry point};
      \draw[link] (message) -- (main);
    }
    \onslide<7->{
      \codeoverlinex{array}
      \node[message] (message) at ($ (array) + (2,2) $) {Parametertypes, arrays};
      \draw[link] (message) -- (array);
    }
    \onslide<8->{
      \codeunderlinex{println}
      \node[message] (message) at ($ (println) + (0,-1) $) {Member access (onzuiver wegens static), methodeoproep};
      \draw[link] (message) -- (println);
    }
    \onslide<9->{
      \codeunderlinex{string}
      \node[message] (message) at ($ (string) + (0,-2) $) {String literals};
      \draw[link] (message) -- (string);
    } 
    \onslide<10->{
      \codeoverlinex{semicolon}
      \node[message] (message) at ($ (semicolon) + (0,1) $) {Random syntax};
      \draw[link] (message) -- (semicolon);
    } 
 \end{tikzpicture}
\end{frame}

\begin{frame}
  \frametitle{Python}
  \code[language=python,font=\small,width=.4\linewidth]{hello-world.py}
  \begin{tikzpicture}[overlay,remember picture,
                      message/.style={opacity=0.75,text opacity=1.0,drop shadow,fill=blue!50,font=\small},
                      link/.style={-latex}]

    \onslide<2->{
      \codeoverlinex{py print}
      \node[message] (message) at ($ (py print) + (0,1) $) {Functies};
      \draw[link] (message) -- (py print);
    }
    \onslide<3->{
      \codeunderlinex{py string}
      \node[message] (message) at ($ (py string) + (0,-1) $) {String literals};
      \draw[link] (message) -- (py string);
    }
 \end{tikzpicture}
\end{frame}

\begin{frame}
  \frametitle{Eenvoud}
  \begin{itemize}
    \item Python is ``gelaagder''
    \item Enkel gebruiken wat je nodig hebt
    \item Nieuwe concepten komen geleidelijk aan bod
    \item Kleine initi\"ele leerdrempel
  \end{itemize}
\end{frame}

\subsection{REPL}

\frame{\tableofcontents[currentsubsection]}

\begin{frame}
  \frametitle{REPL}
  \begin{center}
    \begin{tabular}{l}
      \textbf{R}ead \\
      \textbf{E}val \\
      \textbf{P}rint \\
      \textbf{L}oop
    \end{tabular}
  \end{center}
  \begin{itemize}
    \item Laat interactief programmeren toe
    \item Gemakkelijk om code uit te testen
    \item Geen \texttt{main} methode nodig met sysouts
  \end{itemize}
\end{frame}


\subsection{Toepassingen}

\frame{\tableofcontents[currentsubsection]}

\begin{frame}
  \frametitle{Toepassingen: Java}
  \begin{itemize}
    \item Elke taal heeft zijn specialiteit en doelbpubliel
    \item Java bedoeld voor ``programming in the large''
          \begin{itemize}
            \item Grote teams
            \item 100k+ lijnen code
          \end{itemize}
    \item Statische typering als contracten tussen teamleden
          \begin{itemize}
            \item Als u mij 2 integers geeft, beloof ik u een double terug, tenzij ik u een
                  \texttt{WrongIntegerException} toewerp
          \end{itemize}
    \item Klassen om voor structuur te zorgen
    \item Relatief weinig taalfeatures om kloof tussen goede en slechte programmeurs te vermijden
    \item Heeft veel weg van logge bureaucratie
  \end{itemize}
\end{frame}

\begin{frame}
  \frametitle{Toepassingen: Python}
  \begin{itemize}
    \item Python is ``lean and mean''
    \item Code doorgaans korter dan equivalente Javacode
    \item Speelt vaak ondersteunende rol
    \item Python goed voor kleine en middelgrote projecten
          \begin{itemize}
            \item Wegwerpprogramma's
            \item Automatisatie simpele taken
            \item Editors (bv.~leveleditor voor spel)
          \end{itemize}
    \item Python als scripting taal
          \begin{itemize}
            \item Plugins voor grote apps in Python
            \item Bv.~Gimp, Inkscape
          \end{itemize}
    \item Python niet aangeraden voor grote projecten
          \begin{itemize}
            \item Te weinig taalondersteuning
          \end{itemize}
  \end{itemize}
\end{frame}



%%% Local Variables:
%%% mode: latex
%%% TeX-master: "python"
%%% End:

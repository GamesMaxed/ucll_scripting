\section{Praktische Informatie}

\frame{\tableofcontents[currentsection]}


\begin{frame}
  \frametitle{Python Versies}
  \begin{center}
    \begin{tikzpicture}[version/.style={drop shadow,fill=blue!50,minimum height=0.5cm,minimum width=1cm,font=\small},
                        timeline/.style={gray!50,thick},
                        scale=0.7,transform shape]
      \tikzmath{
        function yeary(\year) {
          return -(\year - 2000)*0.375;
        };
      }
      \foreach[count=\i,evaluate={yeary(\year)} as \y] \year in {2000,...,2016} {
        \draw[timeline] (0,\y) -- ++(8,0) node[at start,font=\tiny,left,black] {\year};
      }

      \foreach[count=\i] \version/\year in {2.0/2000,2.1/2001,2.2/2002,2.3/2003,2.4/2005,2.5/2006,2.6/2008,2.7/2010} {
        \tikzmath{
          real \y;
          \y = yeary(\year);
        }
        \node[version] at (3,\y) {\version};
      }

      \foreach[count=\i] \version/\year in {3.0/2008,3.1/2009,3.2/2011,3.3/2012,3.4/2014,3.5/2015,3.6/2016} {
        \tikzmath{
          real \y;
          \y = yeary(\year);
        }
        \node[version] at (5,\y) {\version};
      }
    \end{tikzpicture}
  \end{center}
  \begin{itemize}
    \item 3.0 voerde breaking changes in
    \item 2.x wordt nog in leven gehouden (legacy code)
    \item Dit vak: Python 3.5 (of beter)
  \end{itemize}
\end{frame}

\begin{frame}
  \frametitle{Benodigde Software}
  \begin{itemize}
    \item \link{https://www.python.org/downloads/release/python-360/}{Python 3.5 of beter}
    \item \link{https://git-scm.com/downloads}{Git}
          \begin{itemize}
            \item Windowsgebruikers: Git Bash!
          \end{itemize}
    \item Editor naar keuze
          \begin{itemize}
            \item \link{https://www.gnu.org/software/emacs/download.html}{Emacs} (voor de durvers)
            \item \link{https://vim.sourceforge.io/download.php}{VIM} (voor de masochisten)
            \item Notepad (voor de minimalisten)
            \item \link{https://notepad-plus-plus.org/}{Notepad++}
            \item \link{https://www.sublimetext.com/}{Sublime Text}
            \item Liefst niet MS Office
            \item \dots
          \end{itemize}
    \item Editor maakt weinig uit: we werken via command line
  \end{itemize}
\end{frame}

\begin{frame}
  \frametitle{Eerste Stappen}
  \begin{itemize}
    \item Cursusmateriaal staat op git repository
    \item Kies een locatie op je PC (bv. \texttt{c:\textbackslash git-repos})
          \begin{itemize}
            \item NIET onder Dropbox/OneDrive/\dots
          \end{itemize}
    \item Open Git Bash in deze directory
    \item Voer in (allemaal op \'e\'en lijn, \textvisiblespace\ staat voor spatie)
          \begin{center} \ttfamily\small
            git\textvisiblespace clone\textvisiblespace \\
            https://bitbucket.org/fvogels/ucll-scripting1617.git\textvisiblespace \\
            scripting
          \end{center}
          Of copy paste onderstaande lijn:
          \begin{center} \ttfamily\tiny
            git clone https://bitbucket.org/fvogels/ucll-scripting1617.git scripting
          \end{center}
    \item Enkel onder MacOS/Linux, in \texttt{scripts} subdirectory
          \begin{center} \ttfamily
            chmod u+x prepare run-tests show-help
          \end{center}
  \end{itemize}
\end{frame}

\begin{frame}
  \frametitle{Testen Configuratie}
  \begin{itemize}
    \item Python versie moet minimum 3.5 tonen
          \begin{center} \ttfamily
            python --version
          \end{center}
    \item Ga in repo-directory
          \begin{center} \ttfamily
            cd scripting
          \end{center}
    \item Ga naar de \texttt{scripts} subdirectory
          \begin{center} \ttfamily
            cd scripts
          \end{center}
    \item Voer in
          \begin{center} \ttfamily
            source setup
          \end{center}
    \item Ga naar \texttt{exercises/basics}
          \begin{center} \ttfamily
            cd ../exercises/basics
          \end{center}
    \item Run de tests
          \begin{center} \ttfamily
            rt
          \end{center}
  \end{itemize}
\end{frame}

\begin{frame}
  \frametitle{Algemeen Gebruik Cursusmateriaal}
  \begin{itemize}
    \item Altijd via Git Bash
    \item Altijd eerst \texttt{source setup} uitvoeren
          \begin{itemize}
            \item Instructies komen later over hoe dit te automatiseren
          \end{itemize}
    \item Tests runnen met \texttt{rt}
          \begin{itemize}
            \item Dit runt de tests in alle subdirectories
          \end{itemize}
    \item Oefeningen staan in directories onder \texttt{exercises}
    \item Oefeningen volgen zelfde patroon
          \begin{itemize}
            \item \texttt{tests.py} bevat de tests
            \item \texttt{solution.py} bevat de oplossingen
            \item \texttt{student.py} is het bestand waarin je zelf moet werken
            \item \texttt{readme.txt} bevat uitleg
          \end{itemize}
  \end{itemize}
\end{frame}

\begin{frame}
  \frametitle{Demo}
  \begin{itemize}
    \item Ga naar \texttt{exercises/basics/00-demo}
    \item Vraag om extra uitleg
          \begin{center} \ttfamily
            h
          \end{center}
    \item Run de tests
          \begin{center} \ttfamily
            rt
          \end{center}
    \item Bekijk de output
    \item Extra output te verkrijgen met \texttt{rt -c}
    \item Pas \texttt{student.py} aan
    \item Run de tests opnieuw
          \begin{center} \ttfamily
            rt
          \end{center}
  \end{itemize}
\end{frame}




%%% Local Variables:
%%% mode: latex
%%% TeX-master: "python"
%%% End:

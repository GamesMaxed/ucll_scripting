\section{Types}

\frame{\tableofcontents[currentsection]}

\begin{frame}
  \frametitle{Op CPU Niveau}
  \begin{center} \ttfamily
    10010101 11001111 11000110 00010110 \\
    10110110 10010101 11000101 11010110 \\
    01100011 10110101 00101011 00101011 \\
    10010101 10111010 01000011 10101111 \\
    01011101 00110110 11101111 01010110 \\
    10111010 01010010 11001010 10110101 \\
    10010101 10100111 01010011 01110100 \\
    01110111 01101010 10110101 01011110 \\
  \end{center}
  \begin{itemize}
    \item Wat stellen deze bits voor?
  \end{itemize}
\end{frame}

\begin{frame}
  \frametitle{Interpretatie van Bits}
  \begin{itemize}
    \item Neem een png bestand
    \item Verander de extensie naar mp3
          \begin{itemize}
            \item Wat gebeurt er bij het afspelen ervan?
          \end{itemize}
    \item Verander de extensie naar txt
          \begin{itemize}
            \item Wat toont notepad?
          \end{itemize}
    \item Verander de extensie naar zip
          \begin{itemize}
            \item Welke bestanden steken erin?
          \end{itemize}
    \item Extensie geeft informatie over inhoud bestand
          \begin{itemize}
            \item Geeft betekenis aan bits
          \end{itemize}
  \end{itemize}
\end{frame}

\begin{frame}
  \frametitle{Types in Assembly}
  \begin{itemize}
    \item Assembly kent geen types
    \item Alles is bits
  \end{itemize}
  \code[language=java,font=\small,extra keywords={bits},title=Als Java Assembly was]{bits.java}
\end{frame}

\begin{frame}
  \frametitle{Typefouten}
  \begin{itemize}
    \item Programmeertalen willen extra ondersteuning bieden
    \item Willen betekenis (= type) toekennen aan bits
          \begin{itemize}
            \item Deze bits vormen een integer
            \item Deze bits vormen een kleur
            \item \dots
          \end{itemize}
    \item Toegelaten operaties hangt af van type
          \begin{itemize}
            \item Men kan een tv niet laten remmen
            \item Men kan een auto niet van zender veranderen
          \end{itemize}
    \item Taal checkt dat operaties enkel uitgevoerd worden op objecten met juist type
  \end{itemize}
\end{frame}

\begin{frame}
  \frametitle{Typesystemen}
  \begin{center}
    \begin{tikzpicture}[node/.style={drop shadow,draw,font=\small,fill=white},
                        static/.style={node,fill=green!50},
                        dynamic/.style={node,fill=red!50},
                        link/.style={-latex}]
      \node[node,font=\bfseries] (root) at (0,5) {Typesystemen};
      \visible<2->{
        \node[static] (static) at ($ (root) + (-3,-1) $) {Statisch};
        \node[dynamic] (dynamic) at ($ (root) + (3,-1) $) {Dynamisch};
        \draw[link] (root) -- (static);
        \draw[link] (root) -- (dynamic);
      }

      \visible<3->{
        \node[static] (java) at ($ (static) + (-90:2) $) {Java};
        \node[static] (cpp) at ($ (static) + (-135:2) $) {C++};
        \node[static] (csharp) at ($ (static) + (-180:2) $) {C$^\sharp$};
        \node[static] (haskell) at ($ (static) + (-225:2) $) {Haskell};
        \node[static] (typescript) at ($ (static) + (-40:2) $) {TypeScript};
        \node[static] (fsharp) at ($ (static) + (90:2) $) {F$\sharp$};

        \draw[link] (static) -- (java);
        \draw[link] (static) -- (cpp);
        \draw[link] (static) -- (csharp);
        \draw[link] (static) -- (haskell);
        \draw[link] (static) -- (typescript);
        \draw[link] (static) -- (fsharp);

        \node[dynamic] (smalltalk) at ($ (dynamic) + (90:2) $) {Smalltalk};
        \node[dynamic] (javascript) at ($ (dynamic) + (45:2) $) {JavaScript};
        \node[dynamic] (ruby) at ($ (dynamic) + (0:2) $) {Ruby};
        \node[dynamic] (lisp) at ($ (dynamic) + (-45:2) $) {Lisp};
        \node[dynamic] (python) at ($ (dynamic) + (-90:2) $) {Python};
        \node[dynamic] (erlang) at ($ (dynamic) + (-135:2) $) {Erlang};

        \draw[link] (dynamic) -- (smalltalk);
        \draw[link] (dynamic) -- (javascript);
        \draw[link] (dynamic) -- (ruby);
        \draw[link] (dynamic) -- (lisp);
        \draw[link] (dynamic) -- (python);
        \draw[link] (dynamic) -- (erlang);
      }
    \end{tikzpicture}
  \end{center}
\end{frame}

\subsection{Statisch Getypeerd}

\frame{\tableofcontents[currentsubsection]}

\begin{frame}
  \frametitle{Statisch Getypeerde Talen}
  \begin{itemize}
    \item Analyseren de code
    \item Weigeren te compileren bij typefouten
    \item Code moet meestal type-annotaties bevatten
  \end{itemize}
  \begin{center}
    \begin{tikzpicture}[every node/.style={fill=blue!50,drop shadow,draw,minimum width=2cm,minimum height=1cm},link/.style={-latex}]
      \node (static analysis) at (0,0) {Type checks};
      \node[anchor=west] (compilation) at ($ (static analysis.east) + (1,0) $) {Vertaling};
      \node[anchor=west] (runtime) at ($ (compilation.east) + (1,0) $) {Uitvoering};

      \draw[link] (static analysis) -- (compilation);
      \draw[link] (compilation) -- (runtime);
    \end{tikzpicture}
  \end{center}
\end{frame}

\begin{frame}
  \frametitle{Voorbeeld}
  \code[language=java,width=.6\linewidth]{Min.java}
  \begin{tikzpicture}[remember picture,overlay,
                      message/.style={opacity=0.75,text opacity=1,fill=red!50,draw,drop shadow},
                      link/.style={-latex}]
    \only<2>{
      \codeunderlinex{return}
      \codeunderlinex{x type}
      \codeunderlinex{y type}
      \node[message] (type annotations) at ($ (x type) + (0,-1) $) {Type annotaties};
      \draw[link] (type annotations.north) -- (x type);
      \draw[link] (type annotations) -| (y type);
      \draw[link] (type annotations) -| (return);
    }

    \only<3>{
     \codeunderlinex{comparison}
      \node[message] (comparable) at ($ (comparison) + (2,-1) $) {Kunnen \texttt{x} en \texttt{y} vergeleken worden?};
      \draw[link] let \p1=(comparison), \p2=(comparable.north) in (\x1,\y2) -- (\p1);
    }

    \only<4>{
      \codeunderlinex{x type}
      \codeunderlinex{y type}
      \node[message] (comparable) at ($ (y type) + (0,-1) $) {Ja, het zijn beide \texttt{int}s};
      \draw[link] let \p1=(x type), \p2=(comparable.north) in (\x1,\y2) -- (\p1);
      \draw[link] let \p1=(y type), \p2=(comparable.north) in (\x1,\y2) -- (\p1);
    }

    \only<5>{
     \codeunderlinex{comparison}
      \node[message] (boolean) at ($ (comparison) + (2,-1) $) {\texttt{if} verwacht \texttt{boolean}, dit is ok};
      \draw[link] let \p1=(comparison), \p2=(boolean.north) in (\x1,\y2) -- (\p1);
    }

    \only<6>{
      \codeunderlinex{return x}
      \node[message] (returnable) at ($ (return x) + (2,-1) $) {Mag \texttt{x} gereturnd worden?};
      \draw[link] let \p1=(return x), \p2=(returnable.north) in (\x1,\y2) -- (\p1);
    }

    \only<7>{
      \codeunderlinex{return}
      \codeunderlinex{x type}
      \node[message] (returnable) at ($ (x type) + (0,-1) $) {Ja, type \texttt{x} en returntype komen overeen};
      \draw[link] let \p1=(return), \p2=(returnable.north) in (\x1,\y2) -- (\p1);
      \draw[link] let \p1=(x type), \p2=(returnable.north) in (\x1,\y2) -- (\p1);
    }

    \only<8>{
      \codeoverlinex{return y}
      \node[message] (returnable) at ($ (return y) + (2,1) $) {Mag \texttt{y} gereturnd worden?};
      \draw[link] let \p1=(return y), \p2=(returnable.south) in (\x1,\y2) -- (\p1);
    }

    \only<9>{
      \codeunderlinex{return}
      \codeunderlinex{y type}
      \node[message] (returnable) at ($ (x type) + (0,-1) $) {Ja, type \texttt{y} en returntype komen overeen};
      \draw[link] let \p1=(return), \p2=(returnable.north) in (\x1,\y2) -- (\p1);
      \draw[link] let \p1=(y type), \p2=(returnable.north) in (\x1,\y2) -- (\p1);
    }
  \end{tikzpicture}
\end{frame}

\begin{frame}
  \frametitle{Statische Typering}
  \structure{Voordelen}
  \begin{itemize}
    \item Fouten worden op voorhand gedetecteerd
    \item Leidt tot snellere code
    \item Garantie op ontbreken van typefouten
    \item H\'e\'el nuttig voor grote teams
  \end{itemize}
  \vskip5mm
  \structure{Nadelen}
  \begin{itemize}
    \item Legt strikte beperkingen op
    \item Leidt vaak tot veel boilerplate code
  \end{itemize}
\end{frame}

\subsection{Dynamisch Getypeerd}

\frame{\tableofcontents[currentsubsection]}

\begin{frame}
  \frametitle{Dynamisch Getypeerde Talen}
  \begin{itemize}
    \item Geen code analyse
    \item Tijdens uitvoering krijgen alle objecten een ``type tag''
    \item Type tag identificeert type
    \item Alle type checks gebeuren tijdens uitvoering
  \end{itemize}
  \begin{center}
    \begin{tikzpicture}[every node/.style={fill=blue!50,drop shadow,draw,minimum width=2cm,minimum height=1cm},link/.style={-latex}]
      \node (translation) at (0,0) {Vertaling};
      \node[anchor=west] (type checks) at ($ (translation.east) + (1,0) $) {Type checks};
      \node[anchor=west] (runtime) at ($ (type checks.east) + (1,0) $) {Uitvoering};

      \draw[link] (translation) -- (type checks);
      \draw[link] (type checks) -- (runtime);
    \end{tikzpicture}
  \end{center}
\end{frame}

\begin{frame}
  \frametitle{Dynamische Typering}
  \structure{Voordelen}
  \begin{itemize}
    \item Geeft veel meer vrijheid
    \item Code herbruikbaarder
    \item Kortere code
    \item Eenvoudiger te begrijpen
  \end{itemize}
  \vskip5mm
  \structure{Nadelen}
  \begin{itemize}
    \item Trager
    \item Fouten pas ontdekt tijdens uitvoering
    \item Afgeraden voor grote projecten
  \end{itemize}
\end{frame}

\subsection{Effici\"entie}

\frame{\tableofcontents[currentsubsection]}

\begin{frame}
  \frametitle{Statisch Getypeerde Talen}
  \code[language=java]{add.java}
  Vertaling:
  \code{add.asm}
\end{frame}

\begin{frame}
  \frametitle{Dynamisch Getypeerde Talen}
  \code[language=python]{add.py}
  Mogelijke vertaling (pseudoassembly):
  \code[font=\tiny,width=.4\linewidth]{add-dynamic.asm}
\end{frame}


%%% Local Variables:
%%% mode: latex
%%% TeX-master: "python"
%%% End:

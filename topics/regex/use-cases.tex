\section{Probleemstelling}

\frame{\tableofcontents[currentsection]}

\begin{frame}
  \frametitle{Probleem \#1}
  \begin{quote}
    Schrijf een functie \texttt{isValidUcllEmailAddress}
    die nagaat of een gegeven string een geldig UCLL emailadres is.
  \end{quote}
  \vskip2mm
  \structure{Patronen}
  \begin{center}\tt
    \nonterminal{letters}.\nonterminal{letters}@student.ucll.be \\
    \nonterminal{letters}.\nonterminal{letters}@ucll.be \\
  \end{center}
  \vskip2mm
  \structure{Voorbeelden}
  \begin{itemize}
    \item {\tt jan.janssens@student.ucll.be} is ok
    \item {\tt peter.peters@gmail.com} is niet ok
  \end{itemize}
\end{frame}

\begin{frame}
  \frametitle{Probleem \#2}
  \begin{quote}
    Schrijf een functie \texttt{parseTime} die
    de uren, minuten, seconden en milliseconden uit
    een string kan halen.
  \end{quote}
  \vskip2mm
  \structure{Patroon}
  \begin{center}\tt
    \nonterminal{2 cijfers}:\nonterminal{2 cijfers}:\nonterminal{2 cijfers}.\nonterminal{3 cijfers}
  \end{center}
  \vskip2mm
  \structure{Voorbeeld}
  \begin{center}
    \texttt{"09:45:11.542"}
  \end{center}
  \begin{itemize}
    \item Uur = \texttt{09}
    \item Minuten = \texttt{45}
    \item Seconden = \texttt{11}
    \item Milliseconden = \texttt{542}
  \end{itemize}
\end{frame}

\begin{frame}
  \frametitle{Probleem \#3}
  \begin{quote}
    Schrijf een functie \texttt{scrapeHtml} die in een gegeven string HTML
    zoekt naar alle hyperlinks.
  \end{quote}
  \vskip2mm
  \structure{Patroon}
  \begin{center}\tt
    <a href="\nonterminal{url}">\nonterminal{tekst}</a>
  \end{center}
\end{frame}

\begin{frame}
  \frametitle{Probleem \#4}
  \begin{quote}
    Scjihrf een fnituce	\texttt{scramble} die de
    lteetrs van elk worod door eklaar hlaat, maar de ersete
    en laastte leettrs op hun paatls laat saatn.
  \end{quote}
  \vskip2mm
  \structure{Patroon}
  \[
    \begin{array}{c}
      \nonterminal{letter}_1\nonterminal{letters}_2\nonterminal{letter}_3 \\
      \downarrow \\
      \nonterminal{letter}_1\texttt{mix}(\nonterminal{letters}_2)\nonterminal{letter}_3
    \end{array}
  \]
\end{frame}

\begin{frame}
  \frametitle{Wat Hebben Ze Gemeenschappelijk?}
  \begin{center}\tt
    \nonterminal{letters}.\nonterminal{letters}@student.ucll.be \\
    \nonterminal{letters}.\nonterminal{letters}@ucll.be \\
  \end{center}
  \begin{center}
    \nonterminal{2 cijfers}:\nonterminal{2 cijfers}:\nonterminal{2 cijfers}.\nonterminal{3 cijfers}
  \end{center}
  \begin{center}\tt
    <a href="\nonterminal{url}">\nonterminal{tekst}</a>
  \end{center}
  \[
    \nonterminal{letter}_1\nonterminal{letters}_2\nonterminal{letter}_3
  \]
  \vskip5mm
  \begin{center}
    \visible<2>{
      \Large
      Telkens zoeken naar tekstpatronen
    }
  \end{center}
\end{frame}

\begin{frame}
  \frametitle{Wat Doet Deze Functie?}
  \code[language=python,font=\tiny,width=.95\linewidth]{check.py}
  \vskip2mm
  \visible<2->{
    \structure{Overeenkomstig Patroon}
    \begin{center} \ttfamily
      BE\it{nn} \it{nnnn} \it{nnnn} \it{nnnn}
    \end{center}
    met $n$ een cijfer.
  }
\end{frame}

\begin{frame}
  \frametitle{Tekstpatronen}
  \begin{itemize}
    \item Algoritme uitschrijven voor elk patroon: veel werk
    \item Moet correct zijn
    \item Moet effici\"ent zijn
    \item Moet leesbaar zijn (intentioneel programmeren)
  \end{itemize}
  \vskip5mm
  \structure{In een ideale wereld\dots}
  \begin{itemize}
    \item We willen enkel het patroon zelf neerschrijven
    \item We willen geen lastige algoritmes implementeren
    \item We willen dat het effici\"ent werkt
  \end{itemize}
\end{frame}

%%% Local Variables:
%%% mode: latex
%%% TeX-master: "regular-expressions"
%%% End:

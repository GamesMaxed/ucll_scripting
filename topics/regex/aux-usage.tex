\section{Gebruik}

\frame{\tableofcontents[currentsection]}

\begin{frame}
  \frametitle{Voorbeeld: E-Mailadres Checken}
  \begin{quote}
    Schrijf een functie \texttt{is\_valid\_ucll\_email\_address}
    die nagaat of een gegeven string een geldig UCLL emailadres is.
  \end{quote}
  \code[language=python,font=\small,width=.9\linewidth]{isValidUcllEmailAddress.py}
\end{frame}

\begin{frame}
  \frametitle{Nagaan of String Voldoet Aan Patroon}
  \structure{Volledige string moet voldoen}
  \begin{center} \ttfamily
    re.fullmatch(\itshape{regex}, \itshape{string})
  \end{center}
  \vskip5mm
  \structure{Prefix moet voldoen}
  \begin{center} \ttfamily
    re.match(\itshape{regex}, \itshape{string})
  \end{center}
  \vskip5mm
  \structure{Substring moet voldoen}
  \begin{center} \ttfamily
    re.search(\itshape{regex}, \itshape{string})
  \end{center}
\end{frame}

\begin{frame}
  \frametitle{Voorbeeld: Tijd Parsen}
  \begin{quote}
    Schrijf een functie \texttt{parse\_time} die
    de uren, minuten, seconden en milliseconden uit
    een string kan halen.
  \end{quote}
  \code[language=python,width=.95\linewidth,font=\small]{parseTime.py}
\end{frame}

\begin{frame}
  \frametitle{Strings Ontleden}
  \begin{itemize}
    \item \texttt{re.search}, \texttt{re.match}, \texttt{re.fullmatch} geven in geval van een match een object \texttt{m} terug
    \item Elke \texttt{(...)} in de regex definieert een groep
    \item Groepen worden van links naar rechts genummerd, beginnend met 1
    \item Opvragen wat in groep staat met {\ttfamily m.groups($n$)}
    \item Groep 0 komt overeen met volledige match
  \end{itemize}
\end{frame}

\begin{frame}
  \frametitle{Voorbeeld: Scrapen}
  \begin{quote}
    Schrijf een functie \texttt{scrape\_html} die in een gegeven string HTML
    zoekt naar alle hyperlinks.
  \end{quote}
  \code[language=python]{scrapeHtml.py}
\end{frame}

\begin{frame}
  \frametitle{Alle Matches Opsommen}
  \begin{center}
    \texttt{re.findall(pattern, str)}
  \end{center}
  \begin{itemize}
    \item Geeft lijst van alle matches
    \item Geen overlappende matches
    \item Geeft lijst van tuples indien meerdere groepen in \texttt{pattern}
  \end{itemize}
  \vskip5mm
  \structure{Voorbeeld}
  \begin{center} \ttfamily
    re.findall("123, 324 3; 142, 21", r"\textbackslash d+")
  \end{center}
  geeft
  \begin{center} \ttfamily
    [ "123", "324", "3", "142", "21" ]
  \end{center}
\end{frame}

\begin{frame}
  \frametitle{Voorbeeld: Substituties}
  \begin{quote}
    Scjihrf een fnituce	\texttt{scramble} die de
    lteetrs van elk worod door eklaar hlaat, maar de ersete
    en laastte leettrs op hun paatls laat saatn.
  \end{quote}
  \code[language=python,font=\scriptsize,width=.95\linewidth]{scramble.py}
\end{frame}

\begin{frame}
  \frametitle{Substituties}
  \begin{center}
    \texttt{re.sub(pattern, func, str)}
  \end{center}
  \begin{itemize}
    \item Zoekt naar alle matches in \texttt{str}
    \item Voor elke gevonden match, roept \texttt{func} op
    \item Het resultaat van deze oproep vervangt de match in \texttt{str}
  \end{itemize}
  \vskip5mm
  \begin{center} \ttfamily
    re.sub(".", lambda m:\ m.group(0) * 2, "hello")
  \end{center}
  geeft
  \begin{center} \ttfamily
    hheelllloo
  \end{center}
\end{frame}



%%% Local Variables:
%%% mode: latex
%%% TeX-master: "regular-expressions"
%%% End:

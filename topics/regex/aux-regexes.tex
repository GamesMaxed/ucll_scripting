\section{Regular Expressions}

\frame{\tableofcontents[currentsection]}

\begin{frame}
  \frametitle{Regular Expressions}
  \begin{itemize}
    \item Oplossing: \emph{Regular Expressions} (regexes) \vskip2mm
    \item Een regex stelt een tekstpatroon voor \vskip2mm
    \item Vormen een DSL (Domain Specific Language)
          \begin{itemize}
            \item Een ubergespecialiseerde mini-taal
            \item Vergelijkbaar met SQL: DSL voor database queries
          \end{itemize} \vskip2mm
    \item Niet Python-specifiek
          \begin{itemize}
            \item Rechtstreeks ingebouwd in Perl, grep, Ruby, \dots
            \item Als libraries in andere talen (Python, Java, C\#, \dots)
            \item Soms lichte syntactische verschillen
          \end{itemize}
  \end{itemize}
\end{frame}

\begin{frame}
  \frametitle{Voorbeeld: Literal Character}
  \begin{quote}
    Ga na of een gegeven string \texttt{str}
    gelijk is aan \texttt{abc}.
  \end{quote}
  \vskip2mm
  \structure{Regex}
  \begin{center}
    \texttt{abc}
  \end{center}
  \vskip2mm
  \structure{Python Code}
  \code{abc.py}
  {\small De \texttt{r} vooraan in \texttt{r"abc"} wordt later uitgelegd.}
\end{frame}

\begin{frame}
  \frametitle{Voorbeeld: Herhaling van Nul of Meer (Kleene Star)}
  \begin{quote}
    Ga na of een gegeven string \texttt{str}
    een opeenvolging is van nul of meer \texttt{a}'s.
  \end{quote}
  \vskip2mm
  \structure{Regex}
  \begin{center}
    \texttt{a*}
  \end{center}
  \vskip2mm
  \structure{Python Code}
  \code{zero-plus-as.py}
  \vskip2mm
  \structure{Voorbeelden}
  \begin{matchexamples}
    \match{}
    \match{a}
    \match{aaa}
    \\[2mm]
    \mismatch{b}
  \end{matchexamples}
\end{frame}

\begin{frame}
  \frametitle{Voorbeeld}
  \begin{quote}
    Ga na of een gegeven string \texttt{str}
    een opeenvolging is van nul of meer \texttt{a}'s gevolgd door nul of meer \texttt{b}'s.
  \end{quote}
  \vskip2mm
  \structure{Regex}
  \begin{center}
    \texttt{a*b*}
  \end{center}
  \vskip2mm
  \structure{Voorbeelden}
  \begin{matchexamples}
    \match{}
    \match{a}
    \match{aaa}
    \match{aaaabbbb}
    \match{bbbb}
    \\[2mm]
    \mismatch{x}
    \mismatch{abba}
  \end{matchexamples}
\end{frame}

\begin{frame}
  \frametitle{Voorbeeld: Herhaling van E\'en of Meer}
  \begin{quote}
    Ga na of een gegeven string \texttt{str}
    een opeenvolging is van \'e\'en of meer \texttt{a}'s.
  \end{quote}
  \vskip2mm
  \structure{Regex}
  \begin{center}
    \texttt{aa*} \hspace{2mm} of \hspace{2mm} \texttt{a+}
  \end{center}
  \vskip2mm
  \structure{Voorbeelden}
  \begin{matchexamples}
    \match{a}
    \match{aaa}
    \match{aaaaaaaa}
    \\[2mm]
    \mismatch{}
    \mismatch{b}
    \mismatch{abba}
  \end{matchexamples}
\end{frame}

\begin{frame}
  \frametitle{Voorbeeld: Groeperen}
  \begin{quote}
    Ga na of een gegeven string \texttt{str}
    een herhaling is \'e\'en of meer keer \texttt{abc}.
  \end{quote}
  \vskip2mm
  \structure{Regex}
  \begin{center}
    \texttt{(abc)+}
  \end{center}
  \structure{Voorbeelden}
  \begin{matchexamples}
    \match{abc}
    \match{abcabc}
    \match{abcabcabc}
    \\[2mm]
    \mismatch{}
    \mismatch{abca}
    \mismatch{abcabjabc}
  \end{matchexamples}
\end{frame}

\begin{frame}
  \frametitle{Voorbeeld: Alternatieven}
  \begin{quote}
    Ga na of een gegeven string \texttt{str}
    gelijk is aan \texttt{abc}, \texttt{xyz} of \texttt{123}.
  \end{quote}
  \vskip2mm
  \structure{Regex}
  \begin{center}
    \texttt{abc|xyz|123}
  \end{center}
  \structure{Voorbeelden}
  \begin{matchexamples}
    \match{abc}
    \match{xyz}
    \match{123}
    \\[2mm]
    \mismatch{ayz}
    \mismatch{abcxyz}
  \end{matchexamples}
\end{frame}

\begin{frame}
  \frametitle{Voorbeeld}
  \begin{quote}
    Is de string een opeenvolging van nul of meer keer \texttt{abc}, \texttt{xyz} of \texttt{123}?
  \end{quote}
  \vskip2mm
  \structure{Regex}
  \begin{center}
    \texttt{(abc|xyz|123)*}
  \end{center}
  \structure{Voorbeelden}
  \begin{matchexamples}
    \match{}
    \match{abc}
    \match{123123123}
    \match{abcxyzabc123}
    \\[2mm]
    \mismatch{abca}
    \mismatch{12312}
  \end{matchexamples}
\end{frame}

\begin{frame}
  \frametitle{Overzicht}
  \begin{center}
    \begin{tabular}{ll}
      \toprule
      \textbf{Beschrijving} & \textbf{Regex} \\
      \midrule
      Nul of meer & \texttt{{\itshape x}*} \\
      Nul of \'e\'en keer & \texttt{{\itshape x}?} \\
      E\'en of meer & \texttt{{\itshape x}+} \\
      $n$ keer & \texttt{{\itshape x}\{$n$\}} \\
      Minimaal $n$ keer & \texttt{{\itshape x}\{$n$,\}} \\
      Maximaal $n$ keer & \texttt{{\itshape x}\{,$n$\}} \\
      $n$ tot $m$ keer & \texttt{{\itshape x}\{$n$,$m$\}} \\
      Alternatieven & \texttt{{\itshape x}|{\itshape y}} \\
      Willekeurig teken & \texttt{.} \\
      \bottomrule
    \end{tabular}
  \end{center}
\end{frame}

\begin{frame}
  \frametitle{Escaping}
  \begin{itemize}
    \item De meeste tekens staan voor zichzelf
          \begin{itemize}
            \item Regex \texttt{"a"} staat voor ``de letter a''
          \end{itemize}
          \vskip2mm
    \item Sommige tekens hebben speciale betekenis
          \begin{itemize}
            \item \texttt{*} staat voor herhaling (0+)
            \item \texttt{+} staat voor herhaling (1+)
            \item \texttt{(} en \texttt{)} groeperen
          \end{itemize}
          \vskip2mm
    \item Indien men een speciaal teken zelf bedoelt, moet men het escapen
  \end{itemize}
\end{frame}

\begin{frame}
  \frametitle{Voorbeeld: Escaping}
  \begin{quote}
    Is de string gelijk aan 3 of meer vraagtekens na elkaar?
  \end{quote}
  \vskip2mm
  \structure{Foute Regex}
  \begin{center} \ttfamily
    ????*
  \end{center}
  \vskip2mm
  \structure{Juiste Regex}
  \begin{center} \ttfamily
    \textbackslash?\textbackslash?\textbackslash?\textbackslash?*
  \end{center}
\end{frame}

\begin{frame}
  \frametitle{Hulpnotaties: Character Class}
  \begin{quote}
    Is de string een enkele lowercase klinker?
  \end{quote}
  \vskip2mm
  \structure{Regex}
  \begin{center} \ttfamily
    (a|e|i|o|u)
  \end{center}
  \vskip2mm
  \structure{Verkorte Notatie}
  \begin{center} \ttfamily
    [aeiou]
  \end{center}
\end{frame}

\begin{frame}
  \frametitle{Hulpnotaties: Range}
  \begin{quote}
    Is de string een lowercase letter?
  \end{quote}
  \vskip2mm
  \structure{Regex}
  \begin{center} \ttfamily
    [abcdefghijklmnopqrstuvwxyz]
  \end{center}
  \vskip2mm
  \structure{Verkorte Notatie}
  \begin{center} \ttfamily
    [a-z]
  \end{center}
\end{frame}

\begin{frame}
  \frametitle{Hulpnotaties: Combinatie van Ranges}
  \begin{quote}
    Is de string een letter of cijfer?
  \end{quote}
  \vskip2mm
  \structure{Regex}
  \begin{center} \ttfamily\scriptsize
    [abcdefghijklmnopqrstuvwxyzABCDEFGHIJKLMNOPQRSTUVWXYZ0123456789]
  \end{center}
  \vskip2mm
  \structure{Verkorte Notatie}
  \begin{center} \ttfamily
    [a-zA-Z0-9]
  \end{center}
\end{frame}

\begin{frame}
  \frametitle{Hulpnotaties: Negatie}
  \begin{quote}
    Bevat de string \'e\'en teken dat dan \emph{geen} klinker is?
  \end{quote}
  \vskip2mm
  \structure{Regex}
  \begin{center} \ttfamily\scriptsize
    [0-9bcdfghjklmnpqrstvwxyzBCDFGHJKLMNPQRSTVWXYZ+-?/\dots]
  \end{center}
  \vskip2mm
  \structure{Verkorte Notatie}
  \begin{center} \ttfamily
    [\^{}aeiou]
  \end{center}
\end{frame}

\begin{frame}
  \frametitle{Overzicht}
  \begin{center}
    \begin{tabular}{ll}
      \toprule
      \textbf{Regex} & \textbf{Korte notatie} \\
      \midrule
      \texttt{a|b|c} & \texttt{[abc]} \\
      \texttt{0|1|2|3|4|5} & \texttt{[0-5]} \\
      Geen cijfer & \texttt{[\^{}0-9]} \\
      \texttt{[a-zA-Z\_]} & \texttt{\textbackslash w} \\
      \texttt{[\^{}a-zA-Z\_]} & \texttt{\textbackslash W} \\
      \texttt{[0-9]} & \texttt{\textbackslash d} \\
      \texttt{[\^{}0-9]} & \texttt{\textbackslash D} \\ 
      Spaties & \texttt{\textbackslash s} \\
      Geen spatie & \texttt{\textbackslash S} \\
      \bottomrule
    \end{tabular}
  \end{center}
\end{frame}

\begin{frame}
  \frametitle{Waarom de \texttt{r"..."}?}
  \begin{quote}
    Ga na of de gegeven string gelijk aan een backslash gevolgd door een n is.
  \end{quote}
  \vskip5mm
  \begin{overprint}
    \onslide<1>
    \structure{Poging \#1}
    \begin{center} \ttfamily
      re.fullmatch("\textbackslash n", str)
    \end{center}
    \begin{enumerate}
      \item Python ziet de string \texttt{"\textbackslash n"}
      \item \texttt{\textbackslash n} herkent Python als symbool voor newline (ASCII: \fbox{10})
      \item De string \texttt{"\textbackslash n"} wordt omgezet naar \texttt{"\fbox{10}"}
      \item De \texttt{search} functie krijgt \texttt{"\fbox{10}"} binnen
      \item \texttt{search} zoekt dus naar een newline, niet naar \texttt{\textbackslash n}
    \end{enumerate}

    \onslide<2>
    \structure{Poging \#2}
    \begin{center} \ttfamily
      re.fullmatch("\textbackslash\textbackslash n", str)
    \end{center}
    \begin{enumerate}
      \item Python ziet de string \texttt{"\textbackslash\textbackslash n"}
      \item De string \texttt{"\textbackslash\textbackslash n"} wordt omgezet naar \texttt{"\textbackslash n"}
      \item De \texttt{search} functie krijgt \texttt{"\textbackslash n"} binnen
      \item In regex-speak staat \texttt{\textbackslash n} voor newline
      \item Intern wordt \texttt{\textbackslash n} dus weer omgezet naar \fbox{10}
      \item Patroon zal weer zoeken naar een newline i.p.v.~\texttt{\textbackslash n}
    \end{enumerate}

    \onslide<3>
    \structure{Poging \#3}
    \begin{center} \ttfamily
      re.fullmatch("\textbackslash\textbackslash\textbackslash\textbackslash n", str)
    \end{center}
    \begin{enumerate}
      \item Python ziet de string \texttt{"\textbackslash\textbackslash\textbackslash\textbackslash n"}
      \item De string \texttt{"\textbackslash\textbackslash\textbackslash\textbackslash n"} wordt omgezet naar \texttt{"\textbackslash\textbackslash n"}
      \item De \texttt{search} functie krijgt \texttt{"\textbackslash\textbackslash n"} binnen
      \item In regex-speak staat \texttt{\textbackslash\textbackslash} voor \texttt{\textbackslash}
      \item Intern wordt \texttt{\textbackslash\textbackslash n} omgezet naar \texttt{\textbackslash n}
      \item Patroon zal nu zoeken naar \texttt{\textbackslash n}, zoals gewenst
    \end{enumerate}
  \end{overprint}
\end{frame}

\begin{frame}
  \frametitle{Raw strings}
  \begin{itemize}
    \item \texttt{\textbackslash\textbackslash\textbackslash\textbackslash n} is wat te veel van het goede
    \item Meeste talen ondersteunen ``rauwe strings''
    \item Zijn strings waarbij elk teken letterlijk wordt opgevat
    \item Bv.~\texttt{\textbackslash n} blijft \texttt{\textbackslash n}
    \item In Python: \texttt{r"..."}
  \end{itemize}
  \begin{center} \ttfamily
    re.search("\textbackslash\textbackslash n", str)
  \end{center}
  \begin{itemize}
    \item Dubbele escape is toch nog nodig: binnen regex-speak betekent \texttt{"\textbackslash n"} nog steeds \fbox{10}
  \end{itemize}
\end{frame}

%%% Local Variables:
%%% mode: latex
%%% TeX-master: "regular-expressions"
%%% End:

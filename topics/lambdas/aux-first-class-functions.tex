\section{First Class Functions}

\frame{\tableofcontents[currentsection]}

\begin{frame}
  \frametitle{Potenti\"ele Nieuwe Lussen}
  \structure{In Voorgaande Voorbeelden}
  \begin{itemize}
    \item \texttt{filter}
    \item \texttt{map}
    \item \texttt{count}
    \item \texttt{all}
    \item \texttt{any}
  \end{itemize}
  \begin{center} \itshape
    Kunnen we  deze lussen zelf implementeren?
  \end{center}
  \visible<2->{
    \begin{center}
      Ja!
    \end{center}
  }
\end{frame}

\begin{frame}
  \frametitle{\texttt{filter} Als Functie: Stapsgewijze Opbouw}
  \begin{overprint}
    \onslide<1>
    \code[language=python3,font=\small]{selectAdults2.py}
    \codeunderline{selectAdults start}{selectAdults end}

    \onslide<2>
    \code[language=python3,font=\small]{selectAdults3.py}
    \codeunderline[name center=selectAdults2 center 1]{selectAdults2 start1}{selectAdults2 end1}
    \codeoverline[name center=selectAdults2 center 2]{selectAdults2 start2}{selectAdults2 end2}
    \tikz[remember picture,overlay] \draw[thick,red,-latex] let \p1=(selectAdults2 center 2), \p2=(selectAdults2 center 1) in (\p1) -- ++(0,0.1) -- ++(2.5,0) |- ($ (\p2) + (0,-0.4) $) -- (\p2);

    \onslide<3>
    \code[language=python3,font=\small]{selectAdults4.py}
    \codeunderline[name center=selectAdults3 center 1]{selectAdults3 start1}{selectAdults3 end1}
    \codeoverline[name center=selectAdults3 center 1b,stroke/.style={thick,blue}]{selectAdults3 start1}{selectAdults3 end1}
    \codeoverline[name center=selectAdults3 center 2]{selectAdults3 start2}{selectAdults3 end2}
    \codeoverline[name center=selectAdults3 center 3,stroke/.style={thick,blue}]{selectAdults3 start3}{selectAdults3 end3}
    \tikz[remember picture,overlay] \draw[thick,red,-latex] let \p1=(selectAdults3 center 2), \p2=(selectAdults3 center 1) in (\p1) -- ++(0,0.5) |- ($ (\p2) + (0,-0.4) $) -- (\p2);
    \tikz[remember picture,overlay] \draw[thick,blue,-latex] let \p1=(selectAdults3 center 3), \p2=(selectAdults3 center 1b) in (\p1) -- ++(0,0.1) -- ++(3,0) |- ($ (\p2) + (0,0.3) $) -- (\p2);
  \end{overprint}
\end{frame}

\begin{frame}
  \frametitle{\texttt{map} Als Functie}
  \begin{overprint}
    \onslide<1>
    \code[language=python3,font=\small]{getAges2.py}
    \codeoverline{getAges2 start}{getAges2 end}

    \onslide<2>
    \code[language=python3,font=\small]{getAges3.py}
    \codeunderline[name center=getAges3 center 1]{getAges3 start1}{getAges3 end1}
    \codeoverline[name center=getAges3 center 2]{getAges3 start2}{getAges3 end2}
    \tikz[remember picture,overlay] \draw[thick,red,-latex] let \p1=(getAges3 center 2), \p2=(getAges3 center 1) in (\p1) -- ++(0,1) |- ($ (\p2) + (0,-0.4) $) -- (\p2);

    \onslide<3>
    \code[language=python3,font=\small]{getAges4.py}
    \codeunderline[name center=getAges4 center 1]{getAges4 start1}{getAges4 end1}
    \codeoverline[name center=getAges4 center 1b,stroke/.style={thick,blue}]{getAges4 start1}{getAges4 end1}
    \codeoverline[name center=getAges4 center 2]{getAges4 start2}{getAges4 end2}
    \codeoverline[name center=getAges4 center 3,stroke/.style={thick,blue}]{getAges4 start3}{getAges4 end3}
    \tikz[remember picture,overlay] \draw[thick,red,-latex] let \p1=(getAges4 center 2), \p2=(getAges4 center 1) in (\p1) |- ($ (\p2) + (0,-0.4) $) -- (\p2);
    \tikz[remember picture,overlay] \draw[thick,blue,-latex] let \p1=(getAges4 center 3), \p2=(getAges4 center 1b) in (\p1) -- ++(0,0.1) -- ++(5,0) |- ($ (\p2) + (0,0.3) $) -- (\p2);
  \end{overprint}
\end{frame}

\begin{frame}
  \frametitle{First Class Functions}
  \begin{itemize}
    \item In Python zijn functies ``first class citizens''
          \begin{itemize}
            \item Mogelijk om functies mee te geven als argument
            \item Mogelijk om functies te returnen
            \item Mogelijk om functies in variabelen te bewaren
          \end{itemize}
    \item Een functie is een object
    \item Operatie \texttt{()} dient om functie op te roepen
  \end{itemize}
  \code[language=python3,font=\small]{first-class-function.py}
\end{frame}

\begin{frame}
  \frametitle{Bestaande Functies}
  \code[language=python]{map.py}
\end{frame}

\begin{frame}
  \frametitle{Bestaande Functies}
  \code[language=python]{filter.py}
\end{frame}

\begin{frame}
  \frametitle{Bestaande Functies}
  \code[language=python]{any.py}
\end{frame}

\begin{frame}
  \frametitle{Bestaande Functies}
  \code[language=python]{all.py}
\end{frame}


%%% Local Variables:
%%% mode: latex
%%% TeX-master: "lambdas"
%%% End:

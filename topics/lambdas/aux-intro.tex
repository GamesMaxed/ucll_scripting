\begin{frame}
  \frametitle{Recept: Versie 1}
  \begin{overprint}
    \onslide<1>
    \code[width=.9\linewidth]{recipe-java.txt}
    \onslide<2>
    \code[width=.9\linewidth]{recipe-java2.txt}
    \onslide<3>
    \code[width=.9\linewidth]{recipe-java3.txt}
  \end{overprint}
\end{frame}

\begin{frame}
  \frametitle{Imperatieve Stijl}
  \begin{itemize}
    \item Imperatieve stijl
    \item Algoritme = lijst van sequenti\"ele stappen
    \item Volgorde uitvoering is van belang
    \item CPU hanteert imperatieve stijl
  \end{itemize}
\end{frame}

\begin{frame}
  \frametitle{Recept: Versie 2}
  \code[width=.9\linewidth]{recipe-haskell.txt}
\end{frame}

\begin{frame}
  \frametitle{Functionele Stijl}
  \begin{itemize}
    \item Functionele stijl
    \item Algoritme = \'e\'en grote uitdrukking
    \item Volgorde uitvoering is niet van belang
    \item Vergelijkbaar met wiskunde
  \end{itemize}
\end{frame}

\begin{frame}
  \frametitle{Gulden Middenweg}
  \code[width=.9\linewidth]{recipe-python.txt}
\end{frame}

\begin{frame}
  \frametitle{Evolutie}
  \begin{itemize}
    \item Evolutie naar combinatie van imperatief en functioneel
    \item Functionele stijl biedt veel voordelen
          \begin{itemize}
            \item Sneller leesbaar
            \item Doorgaans bugvrijer
            \item Beter te multithreaden
          \end{itemize}
    \item Python heeft imperatieve roots met functionele invloeden
  \end{itemize}
\end{frame}

\begin{frame}
  \frametitle{Evolutie}
  \begin{center}
    \begin{tikzpicture}[language/.style={font=\tiny,draw,fill=white,opacity=.75,text opacity=1}]
      \tikzmath{
        real \minyear;
        real \maxyear;
        int \decades;
        \minyear=1950;
        \maxyear=2020;
        \decades=int((\maxyear-\minyear)/10);
        function yeartox(\year) {
          return {(\year - \minyear)/(\maxyear - \minyear) * 10};
        };
      }
      \path[top color=red,bottom color=green] (0,-2) rectangle (10,2);
      \node[anchor=north,white,opacity=.5] at (5,2) {\scshape imperatief};
      \node[anchor=south,white,opacity=.5] at (5,-2) {\scshape functioneel};
      
      \foreach[count=\i,evaluate={\x=10/\decades*(\i-1);}] \year in {\minyear,1960,...,\maxyear} {
        \draw[thin,opacity=.5] (\x,-3) -- (\x,2) node[at start,below,font=\tiny] {\year};
      }

      \foreach[evaluate={\x=yeartox(\year)}] \language/\year/\y in {Fortran/1956/1.5,Haskell/1990/-1.5,C/1972/1.5,C++/1983/1.4,C++11/2011/0.5,Java/1995/1.5,Java8/2014/1.1,C$^\sharp$/2000/1.0,C$^\sharp$3/2007/0.25,Python/1991/0.8,Ruby/1995/0.5,F$^\sharp$/2005/-0.5,Swift/2014/0.2,Scala/2004/-1,Common Lisp/1984/-0.8} {
        \node[language] at (\x,\y) {\language};
      }
      
    \end{tikzpicture}
  \end{center}
\end{frame}

%%% Local Variables:
%%% mode: latex
%%% TeX-master: "lambdas"
%%% End:

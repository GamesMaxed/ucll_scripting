\section{Lambda's}

\frame{\tableofcontents[currentsection]}

\begin{frame}
  \frametitle{Opvragen Velden van \texttt{Person}-objecten}
  \begin{overprint}
    \onslide<1>
    \code[language=python]{map-names.py}
    \onslide<2>
    \code[language=python]{map-ages.py}
    \onslide<3>
    \code[language=python]{map-heights.py}
  \end{overprint}
  \vskip4mm
  \begin{center}
    \begin{tikzpicture}[scale=0.75,transform shape,
                        person/.style={draw,rectangle split,rectangle split parts=2,minimum width=4cm,drop shadow,fill=white},
                        value/.style={draw,minimum width=4cm,fill=white,drop shadow}]
      \pgfdeclarelayer{background}
      \pgfsetlayers{background,main}

      \node[person] (jan) at (0,0) {
        \textbf{Person}
        \nodepart{two}
        \begin{tabular}{r@{$\;=\;$}l}
          name & Jan \\
          age & 18 \\
          height & 186 \\
          weight & 92 \\
        \end{tabular}
      };
      \node[person,anchor=south west] (piet) at ($ (jan.south east) + (1,0) $)  {
        \textbf{Person}
        \nodepart{two}
        \begin{tabular}{r@{$\;=\;$}l}
          name & Piet \\
          age & 19 \\
          height & 173 \\
          weight & 69 \\
        \end{tabular}
      };
      \node[person,anchor=south west] (tom) at ($ (piet.south east) + (1,0) $)  {
        \textbf{Person}
        \nodepart{two}
        \begin{tabular}{r@{$\;=\;$}l}
          name & Tom \\
          age & 23 \\
          height & 182 \\
          weight & 75 \\
        \end{tabular}
      };

      \node[value,anchor=north] (jan value) at ($ (jan.south) + (0,-1) $) {%
        \only<1>{Jan}%
        \only<2>{18}%
        \only<3>{186}%
      };
      \node[value,anchor=north] (piet value) at ($ (piet.south) + (0,-1) $) {%
        \only<1>{Piet}%
        \only<2>{19}%
        \only<3>{173}%
      };
      \node[value,anchor=north] (tom value) at ($ (tom.south) + (0,-1) $) {%
        \only<1>{Tom}%
        \only<2>{23}%
        \only<3>{182}%
      };

      \draw[-latex] (jan.south) -- (jan value.north);
      \draw[-latex] (piet.south) -- (piet value.north);
      \draw[-latex] (tom.south) -- (tom value.north);

      \begin{pgfonlayer}{background}
        \path[fill=red!25] ($ (jan.north west) + (-0.2,0.2) $) rectangle ($ (tom.south east) + (0.2,-0.2) $);
        \path[fill=red!25] ($ (jan value.north west) + (-0.2,0.2) $) rectangle ($ (tom value.south east) + (0.2,-0.2) $);
      \end{pgfonlayer}
    \end{tikzpicture}
  \end{center}
\end{frame}

\begin{frame}
  \frametitle{Hulpfuncties}
  \begin{itemize}
    \item Opvragen naam: functie \texttt{get\_name}
    \item Opvragen leeftijd: functie \texttt{get\_age}
    \item Opvragen lengte: functie \texttt{get\_height}
    \item Opvragen gewicht: functie \texttt{get\_weight}
    \item Telkens nieuwe functie nodig per veld
    \item Veel typwerk (boilerplate code)
  \end{itemize}
\end{frame}

\begin{frame}
  \frametitle{Lambda's}
  \begin{itemize}
    \item Zulke kleine hulpfuncties vaak nodig
    \item Kortere notatie mogelijk
    \item ``Inline functie''
  \end{itemize}
  \vskip5mm
  \begin{overprint}
    \onslide<1-3>
    \code[language=python]{map-names-lambda.py}
    \only<2>{
      \codeunderlinex{getName p}
      \codeoverlinex{lambda p}
      \begin{tikzpicture}[remember picture,overlay]
        \draw[thick,red,latex-] (getName p) -- ++(0,-1) -| (lambda p);
      \end{tikzpicture}
    }
    \only<3>{
      \codeunderlinex{getName body}
      \codeoverlinex{lambda body}
      \begin{tikzpicture}[remember picture,overlay]
        \draw[thick,red,latex-] (getName body) -- ++(0,-1) -| (lambda body);
      \end{tikzpicture}
    }

    \onslide<4>
    \code[language=python]{map-ages-lambda.py}

    \onslide<5>
    \code[language=python]{map-heights-lambda.py}
  \end{overprint}
\end{frame}

\begin{frame}
  \frametitle{Lambda's}
  \begin{center}
    \ttfamily {\bfseries lambda} \nonterminal{args}:\ \nonterminal{body}
  \end{center}
  \begin{itemize}
    \item Body moet \'e\'en expressie zijn
    \item Impliciete \texttt{return}
    \item Stelt een anonieme functie voor
  \end{itemize}
\end{frame}

\begin{frame}
  \frametitle{Lambda's in Andere Talen}
  \begin{itemize}
    \item First class functions en lambda's niet Python-specifiek
          \begin{itemize}
            \item Java sinds versie 8
            \item C++ sinds C++11
            \item C$^\sharp$
            \item JavaScript
            \item dots
          \end{itemize}
    \item Hebben meerdere doeleinden
          \begin{itemize}
            \item Vervangen lussen
            \item Parallellisme
            \item Asynchrone operaties
          \end{itemize}
  \end{itemize}
\end{frame}

%%% Local Variables:
%%% mode: latex
%%% TeX-master: "lambdas"
%%% End:
